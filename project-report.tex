\documentclass[letterpaper,12pt]{report}

\usepackage[a4paper, margin=1in]{geometry} 


\begin{document}

% title page
\begin{titlepage}
    \begin{center}
        \Huge
        \textbf{ETF Prediction and Trajectory Project}

        \LARGE
        \vspace{5pt}
        ECS 171

        \normalsize

        \vspace{10pt}
        Owen Holt, Dylan Lim, etc.. \\

        \vspace{10pt}
        November XX, 2025
        
        \vspace{40pt}
        \textbf{Abstract} \\
        Understanding the response of financial assets to macroeconomic news 
        events such as interest rate announcements is extremely important for 
        investors. \\
        
        - 2-3 sentences that contain a high-level description of how you 
        address the problem (your method) \\
        
        - 2-3 sentences on results \\

        - 1 sentence on broader impact (how this project/advance achieved by this 
        project could influence the field) \\



    \end{center}
\end{titlepage}

% report page 1 begins
\newpage

\large
\textbf{1 Introduction} \\
\normalsize
- 1-2 sentences on general area \\
- 1-2 sentences on specific sub area \\
- 1 paragraph on specific problem and how it has been addressed so far \\
- 1-2 sentences to a paragraph on what is missing from current approaches and 
why this is important \\
- 1 paragraph on what this paper contributes, how we approach this problem,
and results of the approach. should have clear defeinite claims on what we
have achieved relative to what we claimed was missing


\large
\textbf{2 Methods} \\
\normalsize
Divide it into sections that are well-defined around distinct components, algorithms,
sub-problems. Put references whenever you use/step on previously published work. Use sub-
section numbering (3.1, 3.2, etc.). Describe your algorithms and methods fully so that if anyone
wants to reproduce your results, they can do so (if there are many parameters, consider includ-
ing a parameter file as supplementary material).

\large
\textbf{3 Results} \\
\normalsize
This should only describe the results that you have obtained. Sometimes it makes
sense to discuss your interpretation of why this is happening, but this should be rare. This sec-
tion should only present the facts about performance, robustness, and complexity.

\large
\textbf{4 Discussion} \\
\normalsize
This is where you discuss everything that was presented in the Results section. Ex-
plain why the algorithm performed better in X and not in Y. It is ok to succinctly restate strong
result claims (that were included in the previous section), as long as you continue to explain,
even speculate, why this may be the case. For example: "Our algorithm was faster in X% of the
scenarios than what is currently available. The main driver behind this performance boost is
the Y module, which takes advantage of the Z characteristic of the problem. Indeed, ..."

\large
\textbf{5 Conclusion} \\
\normalsize
Finally, in what some papers refer to as "Conclusion," provide 1-2 sentences summarizing the
purpose of this report and 1-2 sentences on what was achieved (these 2-4 sentences are usually
similar to, or restate, what was said in the abstract). Then discuss what remains to be done (the
road ahead), why (the impact to the field), and how you think it can be achieved (future work).
End the report with 1-2 sentences on how the work presented advances the field in the grand
scheme of things

\large
\textbf{6 References} \\
\normalsize
Please include all relevant references to provide a succinct and accurate summary
2
of past work and challenges in the field. Research has found that the more articles you cite, the
more you will be cited as well. For this report you are expected to have 10-20 references.

\large
\textbf{7 Author Contributions} \\
\normalsize
Clearly state what the contributions of each author are. Since you are
working in large teams, this is the only place where I can see who did what.



\end{document}